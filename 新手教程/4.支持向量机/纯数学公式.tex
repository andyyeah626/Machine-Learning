\documentclass{article}
\usepackage{amsmath}
\usepackage{amssymb}
\usepackage{tikz}
\usepackage[UTF8]{ctex} % 添加中文支持
\usetikzlibrary{shapes,arrows,positioning,calc}

\begin{document}
	
	\section*{支持向量机 (SVM) 原理详解}
	
	\subsection*{1. 问题定义}
	给定训练数据集 \( D = \{ (\mathbf{x}_i, y_i) \}_{i=1}^n \),其中:
	\begin{itemize}
		\item \( \mathbf{x}_i \in \mathbb{R}^d \):特征向量
		\item \( y_i \in \{+1, -1\} \):类别标签
	\end{itemize}
	目标是学习一个分离超平面:
	\[
	\mathbf{w}^T \mathbf{x} + b = 0
	\]
	决策函数为:
	\[
	f(\mathbf{x}) = \text{sign}(\mathbf{w}^T \mathbf{x} + b)
	\]
	
	\subsection*{2. 函数间隔与几何间隔}
	\begin{itemize}
		\item \textbf{函数间隔}:\(\hat{\gamma}_i = y_i (\mathbf{w}^T \mathbf{x}_i + b)\)
		\item \textbf{几何间隔}:\(\gamma_i = \dfrac{y_i (\mathbf{w}^T \mathbf{x}_i + b)}{\|\mathbf{w}\|} = \dfrac{\hat{\gamma}_i}{\|\mathbf{w}\|}\)
	\end{itemize}
	
	\subsection*{3. 硬间隔SVM(线性可分)}
	优化问题:
	\[
	\begin{aligned}
		& \min_{\mathbf{w}, b} \frac{1}{2} \|\mathbf{w}\|^2 \\
		& \text{s.t.} \quad y_i (\mathbf{w}^T \mathbf{x}_i + b) \geq 1, \quad i = 1, \dots, n
	\end{aligned}
	\]
	
	\subsection*{4. 软间隔SVM(线性不可分)}
	引入松弛变量 \(\xi_i \geq 0\) 和惩罚参数 \(C > 0\):
	\[
	\begin{aligned}
		& \min_{\mathbf{w}, b, \boldsymbol{\xi}} \frac{1}{2} \|\mathbf{w}\|^2 + C \sum_{i=1}^n \xi_i \\
		& \text{s.t.} \quad y_i (\mathbf{w}^T \mathbf{x}_i + b) \geq 1 - \xi_i, \quad \xi_i \geq 0, \quad i = 1, \dots, n
	\end{aligned}
	\]
	
	\subsection*{5. 对偶问题与核技巧}
	拉格朗日函数:
	\[
	L(\mathbf{w}, b, \boldsymbol{\alpha}, \boldsymbol{\xi}) = \frac{1}{2} \|\mathbf{w}\|^2 + C \sum_{i=1}^n \xi_i - \sum_{i=1}^n \alpha_i [y_i (\mathbf{w}^T \mathbf{x}_i + b) - 1 + \xi_i] - \sum_{i=1}^n \mu_i \xi_i
	\]
	对偶问题:
	\[
	\begin{aligned}
		& \max_{\boldsymbol{\alpha}} \sum_{i=1}^n \alpha_i - \frac{1}{2} \sum_{i=1}^n \sum_{j=1}^n \alpha_i \alpha_j y_i y_j K(\mathbf{x}_i, \mathbf{x}_j) \\
		& \text{s.t.} \quad 0 \leq \alpha_i \leq C, \quad \sum_{i=1}^n \alpha_i y_i = 0, \quad i = 1, \dots, n
	\end{aligned}
	\]
	决策函数:
	\[
	f(\mathbf{x}) = \text{sign}\left( \sum_{i=1}^n \alpha_i y_i K(\mathbf{x}_i, \mathbf{x}) + b \right)
	\]
	
	\subsection*{6. 常用核函数}
	\begin{itemize}
		\item 线性核:\( K(\mathbf{x}_i, \mathbf{x}_j) = \mathbf{x}_i^T \mathbf{x}_j \)
		\item 多项式核:\( K(\mathbf{x}_i, \mathbf{x}_j) = (\gamma \mathbf{x}_i^T \mathbf{x}_j + r)^d \)
		\item RBF核:\( K(\mathbf{x}_i, \mathbf{x}_j) = \exp(-\gamma \|\mathbf{x}_i - \mathbf{x}_j\|^2) \)
		\item Sigmoid核:\( K(\mathbf{x}_i, \mathbf{x}_j) = \tanh(\gamma \mathbf{x}_i^T \mathbf{x}_j + r) \)
	\end{itemize}
	
	\subsection*{7. 支持向量的重要性}
	\begin{itemize}
		\item 支持向量是满足 \( \alpha_i > 0 \) 的样本点
		\item 决策边界仅由支持向量决定
		\item 具有稀疏性的优点
	\end{itemize}
	
\end{document}