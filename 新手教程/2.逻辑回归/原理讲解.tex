\documentclass[UTF8]{ctexart}
\usepackage{amsmath, amssymb}
\usepackage{graphicx}
\usepackage{xcolor}
\usepackage{listings}
\usepackage{hyperref}
\usepackage{booktabs}
\usepackage{array}
\usepackage{float}
\usepackage{enumitem}

% 修复hyperref冲突
\hypersetup{
	unicode=true,
	pdftitle={逻辑回归与梯度下降原理详解},
	pdfauthor={机器学习教程}
}

\title{逻辑回归与梯度下降原理详解}
\author{机器学习教程}
\date{}


\begin{document}
	
	\maketitle
	
	\tableofcontents
	
	\section{逻辑回归识别手写数字教程}
	
	\subsection{代码概览}
	这段代码的目标是:训练一个逻辑回归模型,让它识别 0~9 的手写数字。
	
	\begin{lstlisting}
		import numpy as np
		from sklearn.linear_model import LogisticRegression
		from tensorflow.keras.datasets import mnist
	\end{lstlisting}
	
	\subsubsection{导入库}
	\begin{itemize}
		\item \texttt{numpy}:科学计算库,用于操作矩阵、数组
		\item \texttt{sklearn.linear\_model}:Scikit-Learn 的机器学习模块
		\item \texttt{tensorflow.keras.datasets}:Keras 自带的数据集
	\end{itemize}
	
	\subsection{MNIST 数据集}
	\begin{lstlisting}
		(X_train, y_train), (X_test, y_test) = mnist.load_data()
	\end{lstlisting}
	
	自动下载并加载经典数据集:
	\begin{itemize}
		\item $X_{\text{train}}$:训练图片(60000 张 28×28 像素灰度图)
		\item $y_{\text{train}}$:对应的标签(0~9)
		\item $X_{\text{test}}$:测试图片(10000 张)
		\item $y_{\text{test}}$:测试标签
	\end{itemize}
	
	\begin{table}[H]
		\centering
		\begin{tabular}{cc}
			\toprule
			图片 & 标签 \\
			\midrule
			手写的"3" & 3 \\
			手写的"7" & 7 \\
			\bottomrule
		\end{tabular}
		\caption{MNIST数据集示例}
	\end{table}
	
	\subsection{数据预处理}
	\subsubsection{展平图像}
	\begin{lstlisting}
		X_train = X_train.reshape(-1, 784)
		X_test = X_test.reshape(-1, 784)
	\end{lstlisting}
	
	原始图片是 $28 \times 28$ 像素,展平为 784 维向量。
	
	\subsubsection{归一化到 [0, 1]}
	\begin{lstlisting}
		X_train = X_train.astype('float32') / 255
		X_test = X_test.astype('float32') / 255
	\end{lstlisting}
	
	像素值从 0~255 归一化到 0~1 区间。
	
	\subsection{逻辑回归模型}
	\begin{lstlisting}
		clf = LogisticRegression(penalty="l1", solver="saga", tol=0.1, max_iter=100)
		clf.fit(X_train, y_train)
	\end{lstlisting}
	
	\subsubsection{逻辑回归原理}
	逻辑回归是分类算法,预测样本属于某类的概率。输入图片向量 $x$,模型计算:
	\[
	P(y = k \mid x) = \frac{e^{w_k^T x + b_k}}{\sum_{j=0}^9 e^{w_j^T x + b_j}}
	\]
	这就是 \textbf{Softmax 回归}(逻辑回归的多分类版本)。
	
	其中:
	\begin{itemize}
		\item $w_k$:第 $k$ 个数字的权重向量(784维)
		\item $b_k$:偏置
		\item 输出是每个数字的概率,选概率最大的类别作为预测
	\end{itemize}
	
	\subsubsection{超参数解释}
	\begin{itemize}
		\item \texttt{penalty="l1"}:使用 L1 正则化
		\item \texttt{solver="saga"}:优化算法,适合大数据和 L1
		\item \texttt{tol=0.1}:容忍误差
		\item \texttt{max\_iter=100}:最大迭代次数
	\end{itemize}
	
	\textbf{L1 正则化(稀疏化)}:训练时最小化损失函数:
	\[
	L = -\sum_i \log P(y_i | x_i) + \lambda \sum_j |w_j|
	\]
	第二项是正则项,用来防止过拟合。
	
	\subsection{模型评估}
	\begin{lstlisting}
		score = clf.score(X_test, y_test)
		print("Test score with L1 penalty: %.4f" % score)
	\end{lstlisting}
	
	计算测试集上的准确率:
	\[
	\text{accuracy} = \frac{\text{预测正确的样本数}}{\text{总样本数}}
	\]
	输出约为:\texttt{Test score with L1 penalty: 0.9187}
	
	\subsection{逻辑回归核心数学公式}
	\subsubsection{二分类情况}
	\[
	\hat{y} = \sigma(w^T x + b)
	\]
	其中 Sigmoid 函数:
	\[
	\sigma(z) = \frac{1}{1 + e^{-z}}
	\]
	将任意数转换为 0~1 之间的概率。
	
	\subsubsection{多分类情况(Softmax 函数)}
	\[
	P(y=k|x) = \frac{e^{w_k^T x + b_k}}{\sum_{j} e^{w_j^T x + b_j}}
	\]
	
	\section{逻辑回归数学原理深度解析}
	
	\subsection{逻辑回归的目标}
	逻辑回归要解决的问题是:给定一组特征 $x$,预测它属于某个类别(比如"是否是数字3")的概率。
	
	\subsection{线性组合:$w^T x + b$}
	假设每个像素都是一个特征(总共784个),每个特征都有一个权重参数 $w_i$ 表示重要程度。加权求和:
	\[
	z = w_1 x_1 + w_2 x_2 + \cdots + w_{784} x_{784} + b
	\]
	或用矩阵形式:
	\[
	z = w^T x + b
	\]
	
	\begin{itemize}
		\item $x$:输入(784维向量)
		\item $w$:模型参数(784个数字)
		\item $b$:偏置(bias),相当于"起点调整"
		\item $w^T x$:内积(dot product)
	\end{itemize}
	
	这一步的本质是:把一个高维输入 $x$,通过加权求和压缩成一个实数 $z$。
	
	\subsection{偏置项 $b$ 的作用}
	如果没有 $b$,函数就一定过原点 $(0,0)$,灵活性太低。加上 $b$ 就相当于给整个函数"上下平移"的自由度。
	
	\subsection{从线性模型到概率模型}
	$z = w^T x + b$ 的结果可能是任意实数,但我们希望输出是概率(0~1之间)。引入 \textbf{Sigmoid 函数}:
	\[
	\hat{y} = \sigma(z) = \frac{1}{1 + e^{-z}}
	\]
	这样:
	\begin{itemize}
		\item 当 $z$ 很大时,$\hat{y} \to 1$
		\item 当 $z$ 很小时,$\hat{y} \to 0$
	\end{itemize}
	
	所以:
	\[
	\hat{y} = P(y = 1 | x) = \sigma(w^T x + b)
	\]
	
	\subsection{参数 $w$ 和 $b$ 的学习过程}
	参数不是人工设定的,而是通过训练数据"学出来"的。
	
	\subsubsection{训练数据}
	\[
	(x_1, y_1), (x_2, y_2), \dots, (x_N, y_N)
	\]
	
	\subsubsection{模型预测}
	对每个样本:
	\[
	\hat{y_i} = \sigma(w^T x_i + b)
	\]
	
	\subsubsection{损失函数}
	逻辑回归使用交叉熵损失:
	\[
	L(w, b) = -\frac{1}{N} \sum_{i=1}^{N} [y_i \log(\hat{y_i}) + (1 - y_i) \log(1 - \hat{y_i})]
	\]
	
	\subsubsection{参数优化}
	通过梯度下降最小化损失函数:
	\begin{align}
		w &:= w - \eta \frac{\partial L}{\partial w} \\
		b &:= b - \eta \frac{\partial L}{\partial b}
	\end{align}
	
	\section{梯度下降原理详解}
	
	\subsection{直觉理解}
	损失函数 $L(w, b)$ 告诉我们模型预测的误差。目标是最小化这个误差,就像在山地地形图上找最低点。
	
	\subsection{梯度下降思路}
	站在山上某一点(当前 $w, b$),要走到谷底(最小点),需要沿着最陡的下坡方向走,即梯度的反方向。
	
	\subsection{数学解释}
	梯度是损失函数对参数的偏导数组:
	\[
	\nabla L = \left[\frac{\partial L}{\partial w}, \frac{\partial L}{\partial b}\right]
	\]
	
	梯度下降更新公式:
	\begin{align}
		w_{\text{new}} &= w_{\text{old}} - \eta \frac{\partial L}{\partial w} \\
		b_{\text{new}} &= b_{\text{old}} - \eta \frac{\partial L}{\partial b}
	\end{align}
	
	\subsection{学习率 $\eta$}
	学习率控制每次"下山"的步长:
	\begin{itemize}
		\item 太小:收敛慢,训练时间长
		\item 太大:可能震荡或不收敛
	\end{itemize}
	
	\subsection{梯度下降示例}
	以简单二次函数为例:
	\[
	L(w) = (w - 3)^2
	\]
	求导:
	\[
	\frac{dL}{dw} = 2(w - 3)
	\]
	梯度下降:
	\[
	w := w - \eta \times 2(w - 3)
	\]
	
	\begin{table}[H]
		\centering
		\begin{tabular}{cccc}
			\toprule
			迭代 & 当前 $w$ & 梯度 $2(w-3)$ & 新的 $w$ ($\eta=0.1$) \\
			\midrule
			0 & 0.0 & -6.0 & $0 - 0.1\times(-6) = 0.6$ \\
			1 & 0.6 & -4.8 & $0.6 - 0.1\times(-4.8)=1.08$ \\
			2 & 1.08 & -3.84 & $1.08 - 0.1\times(-3.84)=1.464$ \\
			3 & 1.464 & -3.072 & $1.464 - 0.1\times(-3.072)=1.7712$ \\
			\ldots & \ldots & \ldots & 最终逼近 $w=3$ \\
			\bottomrule
		\end{tabular}
		\caption{梯度下降迭代过程}
	\end{table}
	
	\subsection{逻辑回归中的梯度公式}
	逻辑回归损失函数:
	\[
	L(w, b) = -\frac{1}{N} \sum_i \left[ y_i \log(\hat{y_i}) + (1-y_i)\log(1-\hat{y_i}) \right]
	\]
	其中:
	\[
	\hat{y_i} = \sigma(w^T x_i + b)
	\]
	
	求导结果:
	\begin{align}
		\frac{\partial L}{\partial w} &= \frac{1}{N}\sum_i (\hat{y_i} - y_i)x_i \\
		\frac{\partial L}{\partial b} &= \frac{1}{N}\sum_i (\hat{y_i} - y_i)
	\end{align}
	
	\subsection{核心概念总结}
	\begin{table}[H]
		\centering
		\begin{tabular}{lp{8cm}}
			\toprule
			概念 & 含义 \\
			\midrule
			$L(w, b)$ & 损失函数,越小表示模型越准 \\
			$\frac{\partial L}{\partial w}$ & 告诉我们 $w$ 应该往哪个方向改才能让损失变小 \\
			$\eta$ & 学习率,控制每次改多少 \\
			更新公式 & $w := w - \eta \frac{\partial L}{\partial w}$ \\
			结果 & 经过多次迭代,最终找到让损失最小的 $w, b$ \\
			\bottomrule
		\end{tabular}
		\caption{梯度下降核心概念}
	\end{table}
	
\end{document}